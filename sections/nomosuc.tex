Cryptographic protocols are, well, protocols and therefore, follow
a predefined communication pattern.
Our key innovation is to represent such communication protocols using
\emph{session types}.

\paragraph*{\textbf{Example Protocol}}
As an example, consider the two-phase commitment protocol.
The ideal functionality of this protocol consists of a \emph{sender} $S$
and \emph{receiver} $R$ connected to a trusted third-party, which we
name $\fcomm$.
The protocol initiates with $S$ sending a $\mb{commit}$ message to $\fcomm$
indicating its intent to \emph{commit} to a bit.
Next, $S$ sends this committed bit to $\fcomm$.
After receiving the committed bit, $\fcomm$ sends a $\mb{commit}$ message
to $R$ indicating that a bit has been committed to, but does not reveal
this bit to $R$.
At a later time, $S$ sends an $\mb{open}$ message to $\fcomm$ expressing
that $S$ wishes to reveal the secret bit to $R$.
Receiving this message, $\fcomm$ in turn sends an $\mb{open}$ message
to $R$ followed by this bit.
The protocol concludes with each party (process) terminating.

In the session-typed setting, we use typed channels to connect two
parties. For instance, the channel connected $S$ to $\fcomm$
has the session type $\m{sender}$ defined as
\[
  \mi{stype} \; \m{sender} = \ichoice{\mb{commit} : \m{bit} \product
  \ichoice{\mb{open} : \one}}
\]
The type constructor $\ichoice$ denotes an \emph{internal choice}
(here with only one choice) dictating that $S$ must send the
$\mb{commit}$ message to $\fcomm$.
Next, we use the type constructor $\product$ to denote that $S$
sends a value of type $\m{bit}$ ($\m{bit} \product \ldots$).
We then use the $\ichoiceop$ constructor again enforcing
that $S$ sends $\mb{open}$ to $\fcomm$.
Finally, the type $\one$ denotes termination, indicating that
$S$ will send $\m{close}$ message to $\fcomm$.

Analogously, the channel connecting $R$ and $\fcomm$ has
type $\m{receiver}$ defined as
\[
  \mi{stype} \; \m{receiver} = \echoice{\mb{commit} : 
  \echoice{\mb{open} : \m{bit} \arrow \one}}
\]
Type constructor $\echoice$ represents \emph{external choice}
which is the dual to internal choice.
It prescribes that $R$ must receive a $\mb{commit}$ message from $\fcomm$,
followed by an $\mb{open}$ message (using another $\echoiceop$ constructor).
$R$ must then receive a bit using the $\arrow$ constructor (dual to
$\product$).
Finally, the session terminates as indicated by type $\one$.

Session types are a type discipline for communication-centric programming
based on message passing via channels. Session-typed channels describe and
enforce the protocol of communication among processes. The base system of
session types is derived from a Curry-Howard interpretation of intuitionistic
linear logic~\cite{CairesCONCUR2010}. This chapter focuses on the linear
fragment of SILL~\cite{PfenningFOSSACS2015} that internalizes session-based
concurrency.
Session types were introduced by Honda~\cite{HondaCONCUR1993}.

\begin{figure*}[!ht]
\begin{lstlisting}[basicstyle=\small\ttfamily,frame=single]
stype sender = +{ commit : bit ^ +{ open : 1 } } 

stype receiver = &{ commit : &{ open : bit -> 1 } }

decl F_comm : (S : sender), (R : receiver) |- (F : 1)

proc F <- F_comm S R =
  case S (
    commit => b = recv S ;
              R.commit ;
              case S (
                open => R.open ;
                        send R b ;
                        wait S ; wait R ; close F ) )
\end{lstlisting}

\caption{The $\mathcal{F}_{\msf{comm}}$ commitment ideal functionality in Nomos. The types for the sender and receiver channel define what inputs they can give to the functionality and what messsages are sent from the functionality back to the receiver.}
\label{fig:nomos:commitment}
\end{figure*}


Linear logic~\cite{GirardTCS1987} is a substructural logic that enjoys
\emph{exchange} as its only structural property, i.e., it does not exhibit
\emph{weakening} or \emph{contraction}. As a result, purely linear
propositions can be viewed as resources that must be used \emph{exactly
once} in a proof. Here, I adopt the intuitionistic version of linear logic,
yielding the following sequent
\[
A_1, \ldots, A_n \vdash C
\]
where $A_1, \ldots, A_n$ are linear antecedents, while $C$ is the linear
succedent.

Under the Curry-Howard isomorphism for intuitionistic linear logic,
propositions are related to session types, proofs to processes and cut
reduction in proofs to communication. Appealing to this correspondence,
a process term $P$ is assigned to the above judgment and each hypothesis
as well as the conclusion is labeled with a \emph{channel}:
\[
x_1 : A_1, \ldots, x_n : A_n \vdash P :: (z : C)
\]
The resulting judgment states that process $P$ \emph{provides} a service
of session type $A$ along channel $z$, \emph{using} the services of session
types $A_1, \ldots, A_n$ provided along channels $x_1, \ldots, x_n$
respectively. The assignment of a channel to the conclusion is convenient
because, unlike functions, processes do not evaluate to a value but continue
to communicate along their providing channel once they have been created
until they terminate. For the judgment to be well-formed, all channel names
have to be distinct. The antecedents are often abbreviated to $\D$.

The balance between providing and using a session is established by the
two fundamental rules of the sequent calculus that are independent of all
logical connectives: \emph{cut} and \emph{identity}. Cut states that if
$P$ provides service $A$ along channel $x$, then $Q$ can use the service
along the same channel at the same type. Identity states that a client of
service $A$ can directly provide $A$.
\begin{mathpar}
\infer[\m{cut}]
{\D_1, \D_2 \vdash \ecutna{x}{P_x}{Q_x} :: (z : C)}
{\D_1 \vdash P_x :: (x : A) \qquad \D_2, x : A \vdash Q_x :: (z : C)}
\and
\infer[\m{id}]
{y : A \vdash \fwd{x}{y} :: (x : A)}
{}
\end{mathpar}
Operationally, the process $\ecutna{x}{P_x}{Q_x}$ creates a globally fresh
channel $c$, spawns a new process $[c/x]P_x$ providing along $c$, and
continues as $[c/x]Q_x$. Conversely, the process $\fwd{c}{d}$
\emph{forwards} any message $M$ that arrives along $d$ to $c$ and
vice-versa. Because channels are used linearly, the forwarding process
can then terminate, making sure to apply proper renaming.

The operational semantics are formalized as a system of \emph{multiset
rewriting rules}~\cite{Cervesato2009SSOS}. I introduce semantic objects
$\proc{c}{P}$ and $\msg{c}{M}$ describing process $P$ (or message $M$)
providing service along channel $c$. Remarkably, in this formulation,
a message is just a particular form of process, thereby not requiring any
special rules for typing; it can be typed just as processes. The semantics
rules for $\m{cut}$ and $\m{id}$ are presented below.
\[
\begin{array}{lll}
(\m{cut}C) & \proc{d}{\ecutna{x}{P_x}{Q_x}} \step \proc{c}{[c/x]P_x},
\proc{d}{[c/x]Q_x} & \fresh{c} \\[1ex]
(\m{id}^+C) & \msg{d}{M}, \proc{c}{\fwd{c}{d}} \step
\msg{c}{[c/d]M} \\
(\m{id}^-C) & \proc{c}{\fwd{c}{d}}, \msg{e}{M(c)} \step
\msg{e}{[d/c]M(c)}
\end{array}
\]
Here, I adopt the convention to use $x$, $y$ and $z$ for channel
\emph{variables} and $c$, $d$ and $e$ for \emph{channels}. Channels are
created at runtime and substituted for channel variables in process terms.
In the last rule, $M(c)$ indicates that $c$ must occur in $M$, implying
it is the sole client of $c$.

The Curry-Howard correspondence gives each linear logic connective an
interpretation as a session type. This session type prescribes the kind of
message that must be sent or received along a channel of this type.
Table~\ref{tab:basic-types} summarizes the description of the type
along with the provider action. I follow a detailed description of each
session type operator.
\begin{table}[t]
\centering
\renewcommand{\arraystretch}{1.3}
\begin{tabular}{l|l|l}
\textbf{Type} & \textbf{Provider Action} & \textbf{Session Continuation} \\\hline
$\ichoice{\ell : A_\ell}_{\ell \in L}$ & send label $k \in L$ & $A_k$ \\
$\echoice{\ell : A_\ell}_{\ell \in L}$ & receive and branch on label $k \in L$ & $A_k$ \\
$\one$ & send token $\m{close}$ & \emph{none} \\
$A \tensor B$ & send channel $c : A$ & $B$ \\
$A \lolli B$ & receive channel $c : A$ & $B$
\end{tabular}
\caption{Basic Session Types.  Every provider action has a matching
client action.}
\label{tab:basic-types}
\end{table}

\subsection*{Internal Choice}
A type $A$ is said to describe a \emph{session}, which is a particular
sequence of interactions. As a first type construct, consider \emph{internal
choice} $\ichoice{\ell : A_\ell}_{\ell \in L}$, an $n$-ary labeled
generalization of the linear logic connective $A \oplus B$. A process
that provides $x : \ichoice{\ell : A_\ell}_{\ell \in L}$ can send any label
$k \in L$ along $x$ and then continue by providing $x : A_k$. The
corresponding process is written as $(\esendl{x}{k} \semi P)$, where
$P$ is the continuation that provides $A_k$. This typing is formalized
by the \emph{right rule} $\oplus R$ in linear sequent calculus. The
corresponding client branches on the label received along $x$ as specified
by the \emph{left rule} $\oplus L$.
\begin{mathpar}
\infer[{\oplus}R]
{\D \vdash (\esendl{x}{k} \semi P) ::
	(x : \ichoice{\ell : A_\ell}_{\ell \in L})}
{(k \in L) \qquad \D \vdash P :: (x : A_k)}
\and
\infer[{\oplus}L]
{\D, (x : \ichoice{\ell : A_\ell}_{\ell \in L}) \vdash
	\ecase{x}{\ell}{Q_\ell}_{\ell \in L} :: (z : C)}
{(\forall \ell \in L) \qquad \D, (x : A_\ell) \vdash Q_\ell :: (z : C)}
\end{mathpar}

Operationally, since communication is asynchronous, the process
$(\esendl{c}{k} \semi P)$ sends a message $k$
along $c$ and continues as $P$ without waiting for it to be received.
As a technical device to ensure that consecutive messages on a
channel arrive in order, the sender also creates a fresh continuation
channel $c'$ so that the message $k$ is actually represented as
$(\esendl{c}{k} \semi \fwd{c}{c'})$ (read: send $k$ along $c$ and
continue as $c'$). The provider also substitutes $c'$ for $c$ enforcing
that the next message is sent on $c'$.
\[
\begin{array}{lll}
(\oplus S) & \proc{c}{\esendl{c}{k} \semi P} \step \proc{c'}{[c'/c]P},
\msg{c}{\esendl{c}{k} \semi \fwd{c}{c'}} & \fresh{c'}
\end{array}
\]
When the message $k$ is received along $c$, the client selects branch
$k$ and also substitutes the continuation channel $c'$ for $c$, thereby
ensuring that it receives the next message on $c'$. This implicit
substitution of the continuation channel ensures the ordering of the
messages.
\[
\begin{array}{ll}
(\oplus C) & \msg{c}{\esendl{c}{k} \semi \fwd{c}{c'}},
\proc{d}{\ecase{c}{\ell}{Q_\ell}_{\ell \in L}} \step \proc{d}{[c'/c]Q_k}
\end{array}
\]

\subsection*{External Choice}
The dual of internal choice is \emph{external choice} $\echoice{\ell :
A_\ell}_{\ell \in L}$, which is the $n$-ary labeled generalization of the
linear logic connective $A \with B$. This dual operator simply reverses
the role of the provider and client. The provider process of
$x : \echoice{\ell : A_\ell}_{\ell \in L}$ branches on receiving a label
$k \in L$ (described in $\with R$), while the client sends this label
(described in $\with L$).
\begin{mathpar}
\infer[\with R]
{\D \vdash \ecase{x}{\ell}{P_\ell}_{\ell \in L} ::
	(x : \echoice{\ell : A_\ell}_{\ell \in L})}
{(\forall \ell \in L) \qquad \D \vdash P_\ell :: (x : A_\ell)}
\and
\infer[\with L]
{\D, (x : \echoice{\ell : A_\ell}_{\ell \in L}) \vdash
\esendl{x}{k} \semi Q :: (z : C)}
{\D, (x : A_k) \vdash Q :: (z : C)}
\end{mathpar}

The operational semantics rules are just the inverse of internal choice.
The provider receives the branching label $k$ sent by the provider. Both
processes perform appropriate substitutions to ensure the order of messages
sent and received is preserved.
\[
\begin{array}{lll}
(\with S) & \proc{d}{\esendl{c}{k} \semi Q} \step \msg{c'}{\esendl{c}{k}
\semi \fwd{c'}{c}}, \proc{d}{[c'/c]Q} & \fresh{c'} \\
(\with C) & \proc{c}{\ecase{c}{\ell}{Q_\ell}_{\ell \in L}},
\msg{c'}{\esendl{c}{k} \semi \fwd{c'}{c}} \step \proc{c'}{[c'/c]Q_k}
\end{array}
\]

\subsection*{Higher-Order Channels}
Session types allow channels to be \emph{higher-order}, i.e., channels
can be exchanged over channels. The session type corresponding to the
linear logic connective $A \tensor B$ allows its provider to send a channel
of type $A$ and then continue with providing $B$. The corresponding
process term $(\esendch{x}{w} \semi P)$ describes sending channel $w$
over channel $x$ and continuing with $P$. This typing is provided by
the rule $\tensor R$. The client, on the other hand, receives this channel
using the term $(\erecvch{x}{y} \semi Q)$and binds it to a channel
variable $y$, as described by $\tensor L$.
\begin{mathpar}
\infer[\tensor R]
{\D, (w : A) \vdash (\esendch{x}{w} \semi P) :: (x : A \tensor B)}
{\D \vdash P :: (x : B)}
\and
\infer[\tensor L]
{\D, (x : A \tensor B) \vdash (\erecvch{x}{y} \semi Q) :: (z : C)}
{\D, (y : A), (x : B) \vdash Q :: (z : C)}
\end{mathpar}
\[
\begin{array}{ll}
(\tensor S) & \proc{c}{\esendch{c}{e} \semi P} \step \proc{c'}{[c'/c]P},
\msg{c}{\esendch{c}{e} \semi \fwd{c}{c'}} \qquad \fresh{c'} \\
(\tensor C) & \msg{c}{\esendch{c}{e} \semi \fwd{c}{c'}},
\proc{d}{\erecvch{c}{x} \semi Q} \step \proc{d}{[c', e/c, x]Q}
\end{array}
\]

The lolli ($\lolli$) operator is dual to $\tensor$. The provider and client
invert their roles, i.e., the provider of $x : A \lolli B$ receives a channel
of type $A$ sent by its client.
\begin{mathpar}
\infer[{\lolli} R]
  {\D \vdash (\erecvch{x}{y} \semi P) :: (x : A \lolli B)}
  {\D, (y : A) \vdash P :: (x : B)}
\and
\infer[{\lolli}L]
{\D, (x : A \lolli B), (y : A) \vdash (\esendch{x}{w} \semi Q) :: (z : C)}
{\D, (x : B) \vdash Q :: (z : C)}
\end{mathpar} 
\[
\begin{array}{ll}
({\lolli}S) & \proc{d}{\esendch{c}{e} \semi Q} \step
\msg{c'}{\esendch{c}{e} \semi \fwd{c'}{c}}, \proc{d}{[c'/c]Q} \qquad
\fresh{c'} \\
({\lolli}C) & \proc{c}{\erecvch{c}{x}}, \msg{c'}{\esendch{c}{e} \semi
\fwd{c'}{c}} \step \proc{c'}{[c', d/c, x]P}
\end{array}
\]

\subsection*{Termination}
The type $\one$, the multiplicative unit of linear logic, represents
termination of a process, which (due to linearity) is not allowed to use
any channels.
\begin{mathpar}
\infer[{\one}R]
{\cdot \vdash \eclose{x} :: (x : \one)}
{\mathstrut}
\and
\infer[{\one}L]
{\D, (x : \one) \vdash (\ewait{x} \semi Q) :: (z : C)}
{\D \vdash Q :: (z : C)}
\end{mathpar}
Operationally, a client has to wait for the corresponding closing message,
which has no continuation since the provider terminates.
\[
\begin{array}{ll}
(\one S) & \proc{c}{\eclose{c}} \step \msg{c}{\eclose{c}} \\
(\one C) & \msg{c}{\eclose{c}}, \proc{d}{\ewait{c} \semi Q} \step
\proc{d}{Q}
\end{array}
\]

\subsection*{Process Definitions}
Process definitions have the form $\D \vdash f = P :: (x : A)$ where $f$
is the name of the process and $P$ its definition. All definitions are
collected in a fixed global signature $\Sg$. Also, since process definitions
are mutually recursive, it is required that for every process in the signature
is well-typed w.r.t. $\Sg$, i.e. $\Sg \semi \D \vdash P :: (x : A)$. For
readability of the examples, I break a definition into two declarations,
one providing the type and the other the process definition binding the
variables $x$ and those in $\Omega$ (generally omitting their types):
\[
\begin{array}{l}
\D \vdash f :: (x : A) \\
\procdef{f}{\D}{x} = P
\end{array}
\]
A new instance of a defined process $f$ can be spawned with
the expression
\[
\procdef{f}{\overline{y}}{x} \semi Q
\]
where $\overline{y}$ is a sequence of variables matching the
antecedents $\D$.  The newly spawned process will use all
variables in $\overline{y}$ and provide $x$ to the continuation $Q$.
The operational semantics reduces the spawn to a cut.
\[
\begin{array}{ll}
(\m{def} C) &
\proc{c}{x \leftarrow f \leftarrow \overline{e} \semi Q}
\step \proc{a}{[a/x, \overline{e}/\D]P},
\proc{c}{[a/x]Q}
\quad \fresh{a}
\end{array}
\]
where $\D \vdash f = P :: (x : A) \in \Sg$.
Here I write $\overline{e}/\D$ to denote substitution of the
channels in $\overline{e}$ for the corresponding variables in
$\D$.

Sometimes a process invocation is a \emph{tail call}, written without
a continuation as $\procdef{f}{\overline{y}}{x}$. This is a
short-hand for
$\procdef{f}{\overline{y}}{x'} \semi \fwd{x}{x'}$ for a
fresh variable $x'$, that is, a fresh channel is created and
immediately identified with $x$ (although it is generally implemented
more efficiently).

\subsection*{Recursive Types}
Session types can be naturally extended to include recursive types.
For this purpose I allow (possibly mutually recursive) type
definitions $X = A$ in the signature, where I require $A$ to be
\emph{contractive}~\cite{Gay05acta}.  This means here that $A$ should
not itself be a type name. The type definitions are
\emph{equi-recursive} so $X$ can be silently replaced by $A$ during
type checking, and no explicit rules for recursive types are needed.

\subsection{Examples}
As a first example, consider a stream of bits defined recursively as
\begin{sill}
  $\m{bits} = {\oplus}\{\m{b0} : \m{bits}, \m{b1} : \m{bits}, \m{\$} : \one\}$
\end{sill}
When considering bits as representing natural numbers, the
least significant bit is sent first.  For example, a process
$\mi{six}$ sending the number $6 = (110)_2$ would be
\begin{sill}
$\cdot \vdash \mi{six} :: (x : \m{bits})$ \\
$x \leftarrow \mi{six} = x.\m{b0} \semi x.\m{b1} \semi x.\m{b1} \semi x.\m{\$} \semi \m{close}\; x$
\end{sill}
Executing $\proc{c_0}{\procdefna{\mi{six}}{c_0}}$ yields
(with some fresh channels $c_1, \ldots, c_4$)
\[
\begin{array}{lcl}
\proc{c_0}{\procdefna{\mi{six}}{c_0}}
& \mapsto^* &
\msg{c_4}{\eclose{c_4}}, \\
 & &\msg{c_3}{\esendl{c_3}{\m{\$}} \semi \fwd{c_3}{c_4}}, \\
 & &\msg{c_2}{\esendl{c_2}{\m{b1}} \semi \fwd{c_2}{c_3}}, \\
 & &\msg{c_1}{\esendl{c_1}{\m{b1}} \semi \fwd{c_1}{c_2}}, \\
 & &\msg{c_0}{\esendl{c_0}{\m{b0}} \semi \fwd{c_0}{c_1}}, \\
\end{array}
\]
As a first example of a recursive process definition, consider one
that just copies the incoming bits on to the outgoing bits.
\begin{sill}
$y : \m{bits} \vdash \mi{copy} :: (x : \m{bits})$ \\
$\procdef{\mi{copy}}{y}{x} = $ \\
\quad $\casedef{y}$ \= $(\labdef{\m{b0}}$ \= $\esendl{x}{\m{b0}} \semi
\procdef{\mi{copy}}{y}{x}$ 
\hspace{1em}\=
\% received $\m{b0}$ on $y$, send $\m{b0}$ on $x$, recurse \\
\> $\mid \labdef{\m{b1}}$ \= $\esendl{x}{\m{b1}} \semi
\procdef{\mi{copy}}{y}{x}$ 
\> \% received $\m{b1}$ on $y$, send $\m{b1}$ on $x$, recurse \\
\> $\mid \labdef{\m{\$}}$ \= $\esendl{x}{\m{\$}} \semi
\ewait{y} \semi \eclose{x})$
\> \% received $\m{\$}$ on $y$,
send $\m{\$}$ on $x$, wait on $y$, close $x$
\end{sill}
Note the occurrence of a (recursive) \emph{tail call} to $\mi{copy}$.

A last example: to increment a bit stream turn
$\m{b0}$ to $\m{b1}$ but then forward the remaining bits unchanged
($\fwd{x}{y}$), or turn $\m{b1}$ to $\m{b0}$ but then increment
the remaining stream ($\procdef{\mi{plus1}}{y}{x}$) to
capture the effect of the carry bit.
\begin{sill}
$y : \m{bits} \vdash \mi{plus1} :: (x : \m{bits})$ \\
$\procdef{\mi{plus1}}{y}{x} = $ \\
\quad $\casedef{y}$ \= $(\labdef{\m{b0}}$ \= $\esendl{x}{\m{b1}} \semi
\fwd{x}{y}$ \\
\> $\mid \labdef{\m{b1}} \esendl{x}{\m{b0}} \semi
\procdef{\mi{plus1}}{y}{x}$ \\
\> $\mid \labdef{\m{\$}} \esendl{x}{\m{\$}} \semi \ewait{y} \semi
\eclose{x})$
\end{sill}

\subsection{Preservation and Progress}
The main theorems that exhibit the deep connection between our type
system and the operational semantics are the usual \emph{type
  preservation} and \emph{progress}, sometimes called \emph{session
  fidelity} and \emph{deadlock freedom}, respectively.

So far, I have only described individual processes. However, processes
exist in a \emph{configuration}. A process configuration is a multiset
of semantic objects, $\proc{c}{P}$ and $\msg{c}{M}$, where any
two offered channels are distinct. A key question is how to type these
configurations. Since they consist of both processes and messages, they
both \emph{use} and \emph{provide} a collection of channels.
And even though a configuration is treated as a multiset, typing imposes
a partial order on the processes and messages where a provider of a
channel appears to the left of its client.

A configuration is typed w.r.t. a signature providing the type declaration
of each process.
A signature $\Sg$ is \emph{well formed} if
(a) every type definition $V = A_V$ is \emph{contractive},
and (b) every process definition
$\D \vdash f = P :: (x : A)$ in $\Sg$
is well typed according to the process typing judgment, i.e.
$\Sg \semi \D \vdash P :: (x : A)$.

I use the following judgment to type a configuration.
\[
\Sg \semi \D_1 \vDash \config :: \D_2
\]
It states that $\Sg$ is well-formed
and that the configuration $\config$
uses the channels in the context $\D_1$ and provides
the channels in the context $\D_2$.
\begin{figure}[t]
\begin{mathpar}
\infer[\m{empty}]
{\Sg \semi \D \vDash (\cdot) :: \D}
{}
\and
\infer[\m{compose}]
{\Sg \semi \D_0 \vDash (\config_1 \; \config_2) :: \D_2}
{\Sg \semi \D_0 \vDash \config_1 :: \D_1 \qquad
\Sg \semi \D_1 \vDash \config_2 :: \D_2}
\and
\infer[\m{proc}]
{\Sg \semi \D, \D_1 \vDash \proc{c}{P} :: (\D, (c : A) )}
{\Sg \semi \D_1 \vdash P :: (c : A)}
\and
\infer[\m{msg}]
{\Sg \semi \D, \D_1 \vDash \msg{c}{P} :: (\D, (c : A) )}
{\Sg \semi \D_1 \vDash P :: (c : A)}
\end{mathpar}
\caption{Typing rules for a configuration}
\label{fig:config_typing}
\end{figure}
The configuration typing judgment is defined using
the rules presented in Figure~\ref{fig:config_typing}.
%
The rule $\m{empty}$ defines that an empty configuration
is well-typed. The rule $\m{compose}$
composes two
configurations $\config_1$ and $\config_2$; $\config_1$ provides
service on the channels in $\D_1$ while $\config_2$ uses
the channels in $\D_2$. The $\m{proc}$ rule creates a configuration
out of a single process. Similarly, the $\m{msg}$ rule creates a
configuration out of a single message.

\begin{theorem}[Type Preservation]
\label{thm:preservation}
If $\Sg \semi \D' \vDash \config :: \D$ and $\config \step \dc$,
then $\Sg \semi \D' \vDash \dc :: \D$.
\end{theorem}
\begin{proof}
  By case analysis on the transition rule, applying inversion to the
  given typing derivation, and then assembling a new derivation of
  $\dc$.
\end{proof}

A process or message is said to be \emph{poised} if it is trying to
communicate along the channel that it provides.  A poised process is
comparable to a value in a sequential language. A configuration is
poised if every process or message in the configuration is poised.
Conceptually, this implies that the configuration is trying to communicate
externally, i.e. along one of the channel it provides.
The progress theorem then shows that either a configuration can take a
step or it is poised.  To prove this I show first that the typing
derivation can be rearranged to go strictly from right to left and
then proceed by induction over this particular derivation.

\begin{theorem}[Global Progress]
\label{thm:progress}
\mbox{}
If $\cdot \vDash \config :: \D$ then either
\begin{enumerate}
\item[(i)] $\config \mapsto \dc$ for some $\dc$, or
\item[(ii)] $\config$ is poised.
\end{enumerate}
\end{theorem}
\begin{proof}
By induction on the right-to-left typing of $\config$ so that either
$\config$ is empty (and therefore poised) or
$\config = (\dc\; \proc{c}{P})$ or
$\config = (\dc\; \msg{c}{M})$. By induction hypothesis, $\dc$ can
either take a step (and then so can $\config$), or $\dc$ is poised.  In
the latter case, I
analyze the cases for $P$ and $M$, applying multiple steps of
inversion to show that in each
case either $\config$ can take a step or is poised.
\end{proof}
