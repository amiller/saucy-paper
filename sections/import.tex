The most recent iteration of the Universal Composability framework introduces a new notion of polynomial time: \textit{import}.
The import mechanism allocates to the environment some integer $n$ of units of import which it can consume or send to other ITIs.
When writing to another ITI, $\mathcal{Z}$ can specify an amount of import to sent alongside the message, and the receiving ITI can now use this import accordingly.
A good analogy to make for import is to consider them as tokens that are exchanged between ITIs. 
The definition of a polynomially bounded syste of ITIs now becomes: 

Latex definition: A system of ITIs has a polynomial run time if its execution can be bounded by some polynoial $T(n)$ where $n$ is the total amount of import in the system.



