The most recent iteration of the Universal Composability framework introduces a new notion of polynomial time: \textit{import}.
The import mechanism allocates to the environment some integer $n$ of units of import which it can consume or send to other ITIs.
When writing to another ITI, $\mathcal{Z}$ can specify an amount of import to sent alongside the message, and the receiving ITI can now use this import accordingly.
A good analogy to make for import is to consider them as tokens that are exchanged between ITIs. 
The definition of a polynomially bounded syste of ITIs now becomes: 


\paragraph{Balanced Environments}
The import mechanism above allows the environment give the protocol arbitrarily more import than the adversary.
Such a siuation is unnatural and undesirable. Consider the following argument.

Imagine a protocol $\pi$ and another protocol $\widetilde{\pi}$ which is identical to $\pi$ in every way except that the parties of $\widetilde{\pi}$ first send a message to the adversary proportional in length to their import and expects the adversary to echo the message back. 
Then $\pi$ does not UC-emulate $\widetilde{\pi}$ according to standard UC-emulation for the following reason: $\mathcal{A}$ is an adversary tha delivers all protocol messages and halts. A simulator $\mathcal{S}$ would need sufficient import to handle these messages, but an environment $\mathcal{Z}$ can always give $\widetilde{\pi}$ more import than $\mathcal{S}$ rendering the protocol un-simulatable. 

Such a restrictive definition in clearly unnatural. Therefore, the UC framework introducec an environment constraint: \textit{balanced environments}.

\begin{definition} \label{def:balancedenvironments}
An environment is \msf{balanced} if, at a certain point of execution, it provided import $n_1,...,n_k$ to $k$ ITIs overall, then the overall import of the inputs to the adversary is at least $n_1 + ... + n_k$.
\end{definition}
